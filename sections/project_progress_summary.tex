%===============================================================================
\section{Project Progress Summary}

% !!! Reference back to the project plan !!!

% Provides a concise and informative summary of what has been achieved so far.
% Provide a high-level overview of the project's current status. Briefly mention problematic or delayed work packages, issues, or blockers that require stakeholder attention.


% TEAM LEADER will write this section!! No one is allowed to write here. 


% Overview of work package status 
As of \today, 3 out of 17 work packages have been completed, 9 are in progress\footnote{In progress also entails packages where a sizeable part of the work has been completed, but other parts have been cancelled.}, 0 have not started yet (upcoming), 0 are on hold and 5 have been cancelled. 
% Issues
The project has encountered substantial issues. Amongst the most notable are: components being delivered late or not being delivered at all; problems with software tools and the lack of documentation for the software tools; and the severe lack of time given the scope. All of this has caused most of the progress to fall short of fully achieving the milestones.
% State
The system as a whole is in an unfinished state, but some parts are in an operational state. Evaluating the systems performance against any metrics is not currently feasible. $90\%$ of the allocated budget for the project has been utilized, and the project will be unable to deliver according to the initial plan and requirements.

% What should be prioritized in future
Going forward with the project, it is strongly recommended that focus be placed on: 1.4 \textit{Communication protocol}, the individual parts can not be integrated without this crucial work package; and 1.3.2.3 \textit{Implement a velocity distributor}, to enable the robots to take velocity commands and output \ac{rpm} commands to each individual wheel motor.
% Advice
Furthermore, it is not advised to use STM32CubeIDE and the associated tool MC Workbench because of the buggy and unreliable state that they are in, as well as the poor error messages; nor the PyTorch C++ library because of its lack of documentation.

%===============================================================================

\begin{comment}

\end{comment}

%===============================================================================